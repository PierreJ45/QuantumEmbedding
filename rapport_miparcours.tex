
\documentclass[11pt,a4paper]{article}
\usepackage[utf8]{inputenc}
\usepackage[T1]{fontenc}
\usepackage{lmodern}
\usepackage{geometry}
\usepackage{graphicx}
\usepackage{hyperref}
\usepackage{amsmath}
\usepackage{amsfonts}
\usepackage{amssymb}
\usepackage{enumitem}
\usepackage{titlesec}
\usepackage{fancyhdr}
\usepackage{cite}

% Marges
\geometry{margin=2.5cm}

% En-tête et pied de page
\pagestyle{fancy}
\fancyhf{}
\fancyhead[L]{Projet Quantum Embeddings : Rapport à mi-parcours}
\fancyhead[R]{\thepage}

% Titre principal
\title{\textbf{Projet INFONUM : Rapport à mi-parcours}}
\author{Pierre Jourdin \\
        Pierre El Anati \\
        Malek Bouhadida}
\date{For December 20, 2024}

% Sectionnement
\titleformat{\section}{\normalfont\Large\bfseries}{\thesection.}{1em}{}
\titleformat{\subsection}{\normalfont\large\bfseries}{\thesubsection.}{1em}{}

\begin{document}

\maketitle
\tableofcontents
\newpage

\section{Contexte et problème posé}

Le projet est réalisé en collaboration avec \textbf{CACIB} (Crédit Agricole Corporate and Investment Bank). CACIB souhaite explorer le potentiel de l'\textit{informatique quantique} pour le traitement des données financières.

L'objectif principal de ce projet de veille technologique est de développer des \textbf{algorithmes d'embedding} exploitant l'informatique quantique. Ces algorithmes sont destinés à fournir des représentations efficaces des données financières, dans le but de préparer la banque à l'éventuelle démocratisation des ordinateurs quantiques. En effet, si ces technologies deviennent opérationnelles dans le futur, CACIB souhaite disposer d'une base d'algorithmes prête à être déployée pour en tirer parti immédiatement.

Le problème posé peut être formulé ainsi : \textit{Comment utiliser l'informatique quantique pour générer des embeddings optimisés sur des données financières en anticipant les évolutions technologiques à venir ?}


\section{État de l'art des solutions existantes}

\section{Schéma d'ensemble}

\begin{figure}[h!]
    \centering
    \includegraphics[width=0.8\textwidth]{schema.png}
    \caption{Schéma d'ensemble du projet}
    \label{fig:schema}
\end{figure}

\section{Analyse des points clés et justification des choix \\techniques/scientifiques}

\section{Liste des tâches et état d'avancement}

\begin{itemize}
    \item \textbf{Tâche 1 :} Etudier l'état de l'art -- État d'avancement (80\%)
    \item \textbf{Tâche 2 :} Reproduire les résultats de\cite{lloyd2020quantum} pour se faire la main -- État d'avancement (40\%)
    \item \textbf{Tâche 3 :} Implémenter des modèles sur les données de CACIB -- État d'avancement (0\%)
    \item \textbf{Tâche 4 :} Comparer l'approche quantique avec l'approche classique via des benchmarks -- État d'avancement (0\%)
\end{itemize}

\section{Problèmes rencontrés}

\bibliographystyle{plain}
\bibliography{references}

\end{document}
